\documentclass[
  %solution,
  english
	%german
]{tumteaching}

\usepackage{listings}
\usepackage{paralist}

\usepackage{mathpartir}
\usepackage{stmaryrd}
\usepackage{braket}
\usepackage[inline]{enumitem}
\usepackage{amsmath,amsthm}
\usepackage{multirow}
\usepackage{url}
\usepackage[dvipsnames]{xcolor}
\usepackage{centernot}
\usepackage{multicol}
\usepackage{adjustbox}
\usepackage{ltablex}
\usepackage{comment}

\usepackage{info2}

\usetikzlibrary{decorations.pathmorphing}
\usetikzlibrary{shapes.callouts}

\tikzset{snake it/.style={decorate, decoration=snake}}

\ExplSyntaxOn%
\tl_gset:Nn \g_tumteaching_date_tl {WS~2018/19}
\tl_gset:Nn \g_tumteaching_exsheet_nr_tl {4}
\tl_gset:Nn \g_tumteaching_exsheet_deadline_tl {18.11.2018}
\ExplSyntaxOff%

\newcommand{\objective}[1]{\item \small{#1}}
%\newcommand{\mydottedline}[0]{...}
\newcommand{\mydotline}{\makebox[2.0in]{\dotfill}}
\newcolumntype{C}{>{\centering\arraybackslash}X}


\begin{document}

\verticalline
\begin{disclaimer}{General Information}
	Detailed information about the lecture, tutorials and homework assignments can be found on the lecture website\footnote{\url{https://www.in.tum.de/i02/lehre/wintersemester-1819/vorlesungen/functional-programming-and-verification/}}. Solutions have to be submitted to Moodle\footnote{\url{https://www.moodle.tum.de/course/view.php?id=44932}}. Make sure your uploaded documents are readable. Blurred images will be rejected. Use Piazza\footnote{\url{https://piazza.com/tum.de/fall2018/in0003/home}} to ask questions and discuss with your fellow students.
\end{disclaimer}

\verticalline
\vspace*{-5mm}

\begin{disclaimer}{Simplifications}
	Make sure you simplify terms whenever possible. Overly complex proofs with huge formulas due to lack of any meaningful simplification will not be awarded with full points.
\end{disclaimer}

\verticalline
\vspace*{-5mm}

\begin{disclaimer}{OCaml Setup}
	We are going to start OCaml programming in next week's exercises. Please prepare your machines accordingly and bring them to the sessions. A detailed installation guide will be published soon. Please check the website and Moodle.
\end{disclaimer}

\verticalline

\begin{assignment}[L]{Termination}
In the lecture, you have learned how to prove termination of a MiniJava program. Discuss these questions:
	\begin{enumerate}
		\item How can you decide whether a termination proof is required at all?
		\item What is the basic idea of the termination proof?
		\item How has the program to be modified?
		\item What has to be proven?
		\item How is the loop invariant influenced?
	\end{enumerate}		

	
\end{assignment}

\clearpage
\begin{assignment}[L]{Counter Time}
	Prove that the following program does indeed terminate for all inputs.
	\begin{center}
		\begin{tikzpicture}
			\node[circle,draw](start){Start};
\node[below=of start.south, rectangle, draw, yshift=3mm](read_0){$n = read()$};
\path[->](start) edge [] node [right]{} (read_0);
\node[below=of read_0.south, diamond, draw, yshift=3mm](branch_1_fork){n < 0};
\path[->](read_0) edge [] node [right]{} (branch_1_fork);
\draw let \p1=(branch_1_fork.west) in let \p2=(branch_1_fork.west) in node[coordinate](branch_1_fork_angle_l) at (\x1+-10mm, \y2+0mm) {};
\path[-](branch_1_fork) edge [] node [above]{$no$} (branch_1_fork_angle_l);
\draw let \p1=(branch_1_fork.east) in let \p2=(branch_1_fork.east) in node[coordinate](branch_1_fork_angle_r) at (\x1+10mm, \y2+0mm) {};
\path[-](branch_1_fork) edge [] node [above]{$yes$} (branch_1_fork_angle_r);
\node[below=of branch_1_fork_angle_r.south, rectangle, draw, yshift=3mm](assignment_2){$n = -n$};
\path[->](branch_1_fork_angle_r) edge [] node [right]{} (assignment_2);
\node[below=of branch_1_fork_angle_l.south, coordinate, yshift=3mm](branch_1_join_anchor_l){};
\node[below=of assignment_2.south, coordinate, yshift=3mm](branch_1_join_anchor_r){};
\draw let \p1=(branch_1_join_anchor_l) in let \p2=(branch_1_join_anchor_r) in let \p3=(branch_1_fork) in node[circle,draw](branch_1_join) at (\x3,{min(\y1,\y2)}) {};\draw let \p1=(branch_1_join_anchor_l) in let \p2=(branch_1_join) in node[coordinate](branch_1_join_angle_l) at (\x1+0mm, \y2+0mm) {};
\path[-](branch_1_fork_angle_l) edge [] node [right]{} (branch_1_join_angle_l);
\path[->](branch_1_join_angle_l) edge [] node [right]{} (branch_1_join);
\draw let \p1=(branch_1_join_anchor_r) in let \p2=(branch_1_join) in node[coordinate](branch_1_join_angle_r) at (\x1+0mm, \y2+0mm) {};
\path[-](assignment_2) edge [] node [right]{} (branch_1_join_angle_r);
\path[->](branch_1_join_angle_r) edge [] node [right]{} (branch_1_join);
\node[below=of branch_1_join.south, rectangle, draw, yshift=3mm](assignment_3){$i = 0$};
\path[->](branch_1_join) edge [] node [right]{} (assignment_3);
\node[below=of assignment_3.south, rectangle, draw, yshift=3mm](assignment_4){$k = 0$};
\path[->](assignment_3) edge [] node [right]{} (assignment_4);
\node[below=of assignment_4.south, circle, draw, yshift=3mm](loop_5_join){};
\path[->](assignment_4) edge [] node [right]{} (loop_5_join);
\node[below=of loop_5_join.south, diamond, draw, yshift=3mm](loop_5_fork){i == n};
\path[->](loop_5_join) edge [] node [right]{} (loop_5_fork);
\draw let \p1=(loop_5_fork.west) in let \p2=(loop_5_fork.west) in node[coordinate](loop_5_fork_angle_l) at (\x1+-10mm, \y2+0mm) {};
\path[-](loop_5_fork) edge [] node [above]{$yes$} (loop_5_fork_angle_l);
\draw let \p1=(loop_5_fork.east) in let \p2=(loop_5_fork.east) in node[coordinate](loop_5_fork_angle_r) at (\x1+10mm, \y2+0mm) {};
\path[-](loop_5_fork) edge [] node [above]{$no$} (loop_5_fork_angle_r);
\node[below=of loop_5_fork_angle_r.south, diamond, draw, yshift=3mm](branch_6_fork){k < 5};
\path[->](loop_5_fork_angle_r) edge [] node [right]{} (branch_6_fork);
\draw let \p1=(branch_6_fork.west) in let \p2=(branch_6_fork.west) in node[coordinate](branch_6_fork_angle_l) at (\x1+-10mm, \y2+0mm) {};
\path[-](branch_6_fork) edge [] node [above]{$no$} (branch_6_fork_angle_l);
\draw let \p1=(branch_6_fork.east) in let \p2=(branch_6_fork.east) in node[coordinate](branch_6_fork_angle_r) at (\x1+10mm, \y2+0mm) {};
\path[-](branch_6_fork) edge [] node [above]{$yes$} (branch_6_fork_angle_r);
\node[below=of branch_6_fork_angle_l.south, rectangle, draw, yshift=3mm](assignment_7){$i = i+1$};
\path[->](branch_6_fork_angle_l) edge [] node [right]{} (assignment_7);
\node[below=of assignment_7.south, rectangle, draw, yshift=3mm](assignment_8){$k = 0$};
\path[->](assignment_7) edge [] node [right]{} (assignment_8);
\node[below=of branch_6_fork_angle_r.south, rectangle, draw, yshift=3mm](assignment_9){$k = k+1$};
\path[->](branch_6_fork_angle_r) edge [] node [right]{} (assignment_9);
\node[below=of assignment_8.south, coordinate, yshift=3mm](branch_6_join_anchor_l){};
\node[below=of assignment_9.south, coordinate, yshift=3mm](branch_6_join_anchor_r){};
\draw let \p1=(branch_6_join_anchor_l) in let \p2=(branch_6_join_anchor_r) in let \p3=(branch_6_fork) in node[circle,draw](branch_6_join) at (\x3,{min(\y1,\y2)}) {};\draw let \p1=(branch_6_join_anchor_l) in let \p2=(branch_6_join) in node[coordinate](branch_6_join_angle_l) at (\x1+0mm, \y2+0mm) {};
\path[-](assignment_8) edge [] node [right]{} (branch_6_join_angle_l);
\path[->](branch_6_join_angle_l) edge [] node [right]{} (branch_6_join);
\draw let \p1=(branch_6_join_anchor_r) in let \p2=(branch_6_join) in node[coordinate](branch_6_join_angle_r) at (\x1+0mm, \y2+0mm) {};
\path[-](assignment_9) edge [] node [right]{} (branch_6_join_angle_r);
\path[->](branch_6_join_angle_r) edge [] node [right]{} (branch_6_join);
\draw let \p1=(branch_6_join.south) in let \p2=(branch_6_join.south) in node[coordinate](loop_5_back_edge_angle1) at (\x1+0mm, \y2+-10mm) {};
\path[-](branch_6_join) edge [] node [right]{} (loop_5_back_edge_angle1);
\draw let \p1=(loop_5_back_edge_angle1) in let \p2=(loop_5_back_edge_angle1) in node[coordinate](loop_5_back_edge_angle2) at (\x1+35mm, \y2+0mm) {};
\path[-](loop_5_back_edge_angle1) edge [] node [right]{} (loop_5_back_edge_angle2);
\draw let \p1=(loop_5_back_edge_angle2) in let \p2=(loop_5_join) in node[coordinate](loop_5_back_edge_angle3) at (\x1+0mm, \y2+0mm) {};
\path[-](loop_5_back_edge_angle2) edge [] node [right]{} (loop_5_back_edge_angle3);
\path[->](loop_5_back_edge_angle3) edge [] node [right]{} (loop_5_join);
\node[below=of loop_5_fork_angle_l.south, circle, draw,yshift=3mm](stop){Stop};
\path[->](loop_5_fork_angle_l) edge [] node [left]{} (stop);

		\end{tikzpicture}
	\end{center}

	\clearpage
\end{assignment}

\clearpage
\begin{assignment}[L]{A Wavy Approach}
	Prove termination of the following program:
	\begin{center}
		\begin{tikzpicture}
			\node[circle,draw](start){Start};
\node[below=of start.south, rectangle, draw, yshift=3mm](read_0){$a = read()$};
\path[->](start) edge [] node [right]{} (read_0);
\node[below=of read_0.south, circle, draw, yshift=3mm](loop_1_join){};
\path[->](read_0) edge [] node [right]{} (loop_1_join);
\node[below=of loop_1_join.south, diamond, draw, yshift=3mm](loop_1_fork){a == 0};
\path[->](loop_1_join) edge [] node [right]{} (loop_1_fork);
\draw let \p1=(loop_1_fork.west) in let \p2=(loop_1_fork.west) in node[coordinate](loop_1_fork_angle_l) at (\x1+-10mm, \y2+0mm) {};
\path[-](loop_1_fork) edge [] node [above]{$yes$} (loop_1_fork_angle_l);
\draw let \p1=(loop_1_fork.east) in let \p2=(loop_1_fork.east) in node[coordinate](loop_1_fork_angle_r) at (\x1+10mm, \y2+0mm) {};
\path[-](loop_1_fork) edge [] node [above]{$no$} (loop_1_fork_angle_r);
\node[below=of loop_1_fork_angle_r.south, diamond, draw, yshift=3mm](branch_2_fork){a < 0};
\path[->](loop_1_fork_angle_r) edge [] node [right]{} (branch_2_fork);
\draw let \p1=(branch_2_fork.west) in let \p2=(branch_2_fork.west) in node[coordinate](branch_2_fork_angle_l) at (\x1+-10mm, \y2+0mm) {};
\path[-](branch_2_fork) edge [] node [above]{$no$} (branch_2_fork_angle_l);
\draw let \p1=(branch_2_fork.east) in let \p2=(branch_2_fork.east) in node[coordinate](branch_2_fork_angle_r) at (\x1+10mm, \y2+0mm) {};
\path[-](branch_2_fork) edge [] node [above]{$yes$} (branch_2_fork_angle_r);
\node[below=of branch_2_fork_angle_l.south, rectangle, draw, yshift=3mm](assignment_3){$a = -a + 1$};
\path[->](branch_2_fork_angle_l) edge [] node [right]{} (assignment_3);
\node[below=of branch_2_fork_angle_r.south, rectangle, draw, yshift=3mm](assignment_4){$a = -a -1$};
\path[->](branch_2_fork_angle_r) edge [] node [right]{} (assignment_4);
\node[below=of assignment_3.south, coordinate, yshift=3mm](branch_2_join_anchor_l){};
\node[below=of assignment_4.south, coordinate, yshift=3mm](branch_2_join_anchor_r){};
\draw let \p1=(branch_2_join_anchor_l) in let \p2=(branch_2_join_anchor_r) in let \p3=(branch_2_fork) in node[circle,draw](branch_2_join) at (\x3,{min(\y1,\y2)}) {};\draw let \p1=(branch_2_join_anchor_l) in let \p2=(branch_2_join) in node[coordinate](branch_2_join_angle_l) at (\x1+0mm, \y2+0mm) {};
\path[-](assignment_3) edge [] node [right]{} (branch_2_join_angle_l);
\path[->](branch_2_join_angle_l) edge [] node [right]{} (branch_2_join);
\draw let \p1=(branch_2_join_anchor_r) in let \p2=(branch_2_join) in node[coordinate](branch_2_join_angle_r) at (\x1+0mm, \y2+0mm) {};
\path[-](assignment_4) edge [] node [right]{} (branch_2_join_angle_r);
\path[->](branch_2_join_angle_r) edge [] node [right]{} (branch_2_join);
\draw let \p1=(branch_2_join.south) in let \p2=(branch_2_join.south) in node[coordinate](loop_1_back_edge_angle1) at (\x1+0mm, \y2+-10mm) {};
\path[-](branch_2_join) edge [] node [right]{} (loop_1_back_edge_angle1);
\draw let \p1=(loop_1_back_edge_angle1) in let \p2=(loop_1_back_edge_angle1) in node[coordinate](loop_1_back_edge_angle2) at (\x1+40mm, \y2+0mm) {};
\path[-](loop_1_back_edge_angle1) edge [] node [right]{} (loop_1_back_edge_angle2);
\draw let \p1=(loop_1_back_edge_angle2) in let \p2=(loop_1_join) in node[coordinate](loop_1_back_edge_angle3) at (\x1+0mm, \y2+0mm) {};
\path[-](loop_1_back_edge_angle2) edge [] node [right]{} (loop_1_back_edge_angle3);
\path[->](loop_1_back_edge_angle3) edge [] node [right]{} (loop_1_join);
\node[below=of loop_1_fork_angle_l.south, circle, draw,yshift=3mm](stop){Stop};
\path[->](loop_1_fork_angle_l) edge [] node [left]{} (stop);

		\end{tikzpicture}
	\end{center}

\end{assignment}

\clearpage
\begin{assignment}[H,points=6]{Counting Twelves}
	Prove termination of the following program:
	\begin{center}
		\begin{tikzpicture}
			\node[circle,draw](start){Start};
\node[below=of start.south, rectangle, draw, yshift=3mm](assignment_0){$i = 0$};
\path[->](start) edge [] node [right]{} (assignment_0);
\node[below=of assignment_0.south, rectangle, draw, yshift=3mm](read_1){$n = read()$};
\path[->](assignment_0) edge [] node [right]{} (read_1);
\node[below=of read_1.south, rectangle, draw, yshift=3mm](assignment_2){$k = 0$};
\path[->](read_1) edge [] node [right]{} (assignment_2);
\node[below=of assignment_2.south, circle, draw, yshift=3mm](loop_3_join){};
\path[->](assignment_2) edge [] node [right]{} (loop_3_join);
\node[below=of loop_3_join.south, diamond, draw, yshift=3mm](loop_3_fork){i == n*n};
\path[->](loop_3_join) edge [] node [right]{} (loop_3_fork);
\draw let \p1=(loop_3_fork.west) in let \p2=(loop_3_fork.west) in node[coordinate](loop_3_fork_angle_l) at (\x1+-10mm, \y2+0mm) {};
\path[-](loop_3_fork) edge [] node [above]{$yes$} (loop_3_fork_angle_l);
\draw let \p1=(loop_3_fork.east) in let \p2=(loop_3_fork.east) in node[coordinate](loop_3_fork_angle_r) at (\x1+10mm, \y2+0mm) {};
\path[-](loop_3_fork) edge [] node [above]{$no$} (loop_3_fork_angle_r);
\node[below=of loop_3_fork_angle_r.south, rectangle, draw, yshift=3mm](assignment_4){$k = k+1$};
\path[->](loop_3_fork_angle_r) edge [] node [right]{} (assignment_4);
\node[below=of assignment_4.south, diamond, draw, yshift=3mm](branch_5_fork){k == 12};
\path[->](assignment_4) edge [] node [right]{} (branch_5_fork);
\draw let \p1=(branch_5_fork.west) in let \p2=(branch_5_fork.west) in node[coordinate](branch_5_fork_angle_l) at (\x1+-10mm, \y2+0mm) {};
\path[-](branch_5_fork) edge [] node [above]{$no$} (branch_5_fork_angle_l);
\draw let \p1=(branch_5_fork.east) in let \p2=(branch_5_fork.east) in node[coordinate](branch_5_fork_angle_r) at (\x1+10mm, \y2+0mm) {};
\path[-](branch_5_fork) edge [] node [above]{$yes$} (branch_5_fork_angle_r);
\node[below=of branch_5_fork_angle_r.south, rectangle, draw, yshift=3mm](assignment_6){$i = i+1$};
\path[->](branch_5_fork_angle_r) edge [] node [right]{} (assignment_6);
\node[below=of assignment_6.south, rectangle, draw, yshift=3mm](assignment_7){$k = 0$};
\path[->](assignment_6) edge [] node [right]{} (assignment_7);
\node[below=of branch_5_fork_angle_l.south, coordinate, yshift=3mm](branch_5_join_anchor_l){};
\node[below=of assignment_7.south, coordinate, yshift=3mm](branch_5_join_anchor_r){};
\draw let \p1=(branch_5_join_anchor_l) in let \p2=(branch_5_join_anchor_r) in let \p3=(branch_5_fork) in node[circle,draw](branch_5_join) at (\x3,{min(\y1,\y2)}) {};\draw let \p1=(branch_5_join_anchor_l) in let \p2=(branch_5_join) in node[coordinate](branch_5_join_angle_l) at (\x1+0mm, \y2+0mm) {};
\path[-](branch_5_fork_angle_l) edge [] node [right]{} (branch_5_join_angle_l);
\path[->](branch_5_join_angle_l) edge [] node [right]{} (branch_5_join);
\draw let \p1=(branch_5_join_anchor_r) in let \p2=(branch_5_join) in node[coordinate](branch_5_join_angle_r) at (\x1+0mm, \y2+0mm) {};
\path[-](assignment_7) edge [] node [right]{} (branch_5_join_angle_r);
\path[->](branch_5_join_angle_r) edge [] node [right]{} (branch_5_join);
\draw let \p1=(branch_5_join.south) in let \p2=(branch_5_join.south) in node[coordinate](loop_3_back_edge_angle1) at (\x1+0mm, \y2+-10mm) {};
\path[-](branch_5_join) edge [] node [right]{} (loop_3_back_edge_angle1);
\draw let \p1=(loop_3_back_edge_angle1) in let \p2=(loop_3_back_edge_angle1) in node[coordinate](loop_3_back_edge_angle2) at (\x1+40mm, \y2+0mm) {};
\path[-](loop_3_back_edge_angle1) edge [] node [right]{} (loop_3_back_edge_angle2);
\draw let \p1=(loop_3_back_edge_angle2) in let \p2=(loop_3_join) in node[coordinate](loop_3_back_edge_angle3) at (\x1+0mm, \y2+0mm) {};
\path[-](loop_3_back_edge_angle2) edge [] node [right]{} (loop_3_back_edge_angle3);
\path[->](loop_3_back_edge_angle3) edge [] node [right]{} (loop_3_join);
\node[below=of loop_3_fork_angle_l.south, circle, draw,yshift=3mm](stop){Stop};
\path[->](loop_3_fork_angle_l) edge [] node [left]{} (stop);

		\end{tikzpicture}
	\end{center}

\end{assignment}

\clearpage
\begin{assignment}[H,points=4]{aaaaaaaaaab}
	Prove termination of the following program:
	\begin{center}
		\begin{tikzpicture}
			\node[circle,draw](start){Start};
\node[below=of start.south, rectangle, draw, yshift=3mm](read_0){$n = read()$};
\path[->](start) edge [] node [right]{} (read_0);
\node[below=of read_0.south, rectangle, draw, yshift=3mm](assignment_1){$a = n*n$};
\path[->](read_0) edge [] node [right]{} (assignment_1);
\node[below=of assignment_1.south, rectangle, draw, yshift=3mm](assignment_2){$b = 1$};
\path[->](assignment_1) edge [] node [right]{} (assignment_2);
\node[below=of assignment_2.south, circle, draw, yshift=3mm](loop_4_join){};
\path[->](assignment_2) edge [] node [right]{} (loop_4_join);
\node[below=of loop_4_join.south, diamond, draw, yshift=3mm](loop_4_fork){b == 1};
\path[->](loop_4_join) edge [] node [left,yshift=6mm,xshift=-2mm]{} (loop_4_fork);
\draw let \p1=(loop_4_fork.west) in let \p2=(loop_4_fork.west) in node[coordinate](loop_4_fork_angle_l) at (\x1+-10mm, \y2+0mm) {};
\path[-](loop_4_fork) edge [] node [above]{$no$} (loop_4_fork_angle_l);
\draw let \p1=(loop_4_fork.east) in let \p2=(loop_4_fork.east) in node[coordinate](loop_4_fork_angle_r) at (\x1+10mm, \y2+0mm) {};
\path[-](loop_4_fork) edge [] node [above]{$yes$} (loop_4_fork_angle_r);
\node[below=of loop_4_fork_angle_r.south, diamond, draw, yshift=-3mm](branch_5_fork){a > 0};
\path[->](loop_4_fork_angle_r) edge [] node [right]{} (branch_5_fork);
\draw let \p1=(branch_5_fork.west) in let \p2=(branch_5_fork.west) in node[coordinate](branch_5_fork_angle_l) at (\x1+-10mm, \y2+0mm) {};
\path[-](branch_5_fork) edge [] node [above]{$no$} (branch_5_fork_angle_l);
\draw let \p1=(branch_5_fork.east) in let \p2=(branch_5_fork.east) in node[coordinate](branch_5_fork_angle_r) at (\x1+10mm, \y2+0mm) {};
\path[-](branch_5_fork) edge [] node [above]{$yes$} (branch_5_fork_angle_r);
\node[below=of branch_5_fork_angle_l.south, rectangle, draw, yshift=3mm](assignment_6){$b = a$};
\path[->](branch_5_fork_angle_l) edge [] node [left]{} (assignment_6);
\node[below=of branch_5_fork_angle_r.south, rectangle, draw, yshift=3mm](assignment_7){$a = a-1$};
\path[->](branch_5_fork_angle_r) edge [] node [right]{} (assignment_7);
\node[below=of assignment_6.south, coordinate, yshift=3mm](branch_5_join_anchor_l){};
\node[below=of assignment_7.south, coordinate, yshift=3mm](branch_5_join_anchor_r){};
\draw let \p1=(branch_5_join_anchor_l) in let \p2=(branch_5_join_anchor_r) in let \p3=(branch_5_fork) in node[circle,draw](branch_5_join) at (\x3,{min(\y1,\y2)}) {};\draw let \p1=(branch_5_join_anchor_l) in let \p2=(branch_5_join) in node[coordinate](branch_5_join_angle_l) at (\x1+0mm, \y2+0mm) {};
\path[-](assignment_6) edge [] node [right]{} (branch_5_join_angle_l);
\path[->](branch_5_join_angle_l) edge [] node [right]{} (branch_5_join);
\draw let \p1=(branch_5_join_anchor_r) in let \p2=(branch_5_join) in node[coordinate](branch_5_join_angle_r) at (\x1+0mm, \y2+0mm) {};
\path[-](assignment_7) edge [] node [right]{} (branch_5_join_angle_r);
\path[->](branch_5_join_angle_r) edge [] node [right]{} (branch_5_join);
\node[coordinate,below=of branch_5_join.south,yshift=3mm](loop_4_back_edge_angle1) {};
\path[-](branch_5_join) edge [] node [right]{} (loop_4_back_edge_angle1);
\draw let \p1=(loop_4_back_edge_angle1) in let \p2=(loop_4_back_edge_angle1) in node[coordinate](loop_4_back_edge_angle2) at (\x1+40mm, \y2+0mm) {};
\path[-](loop_4_back_edge_angle1) edge [] node [right]{} (loop_4_back_edge_angle2);
\draw let \p1=(loop_4_back_edge_angle2) in let \p2=(loop_4_join) in node[coordinate](loop_4_back_edge_angle3) at (\x1+0mm, \y2+0mm) {};
\path[-](loop_4_back_edge_angle2) edge [] node [right]{} (loop_4_back_edge_angle3);
\path[->](loop_4_back_edge_angle3) edge [] node [right]{} (loop_4_join);
\node[below=of loop_4_fork_angle_l.south, circle, draw,yshift=3mm](stop){Stop};
\path[->](loop_4_fork_angle_l) edge [] node [left]{} (stop);

		\end{tikzpicture}
	\end{center}

\end{assignment}

\clearpage
\begin{assignment}[H,points=10]{Term-ination}
	Prove that the following program terminates for all inputs:
	\begin{center}
		\begin{tikzpicture}
			\node[circle,draw](start){Start};
\node[below=of start.south, rectangle, draw, yshift=3mm](read_0){$y = read()$};
\path[->](start) edge [] node [right]{} (read_0);
\node[below=of read_0.south, rectangle, draw, yshift=3mm](assignment_1){$x = 0$};
\path[->](read_0) edge [] node [right]{} (assignment_1);
\node[below=of assignment_1.south, rectangle, draw, yshift=3mm](assignment_2){$a = 0$};
\path[->](assignment_1) edge [] node [right]{} (assignment_2);
\node[below=of assignment_2.south, rectangle, draw, yshift=3mm](assignment_3){$b = 0$};
\path[->](assignment_2) edge [] node [right]{} (assignment_3);
\node[below=of assignment_3.south, circle, draw, yshift=3mm](loop_4_join){};
\path[->](assignment_3) edge [] node [right]{} (loop_4_join);
\node[below=of loop_4_join.south, diamond, draw, yshift=3mm](loop_4_fork){x < y};
\path[->](loop_4_join) edge [] node [right]{} (loop_4_fork);
\draw let \p1=(loop_4_fork.west) in let \p2=(loop_4_fork.west) in node[coordinate](loop_4_fork_angle_l) at (\x1+-10mm, \y2+0mm) {};
\path[-](loop_4_fork) edge [] node [above]{$no$} (loop_4_fork_angle_l);
\draw let \p1=(loop_4_fork.east) in let \p2=(loop_4_fork.east) in node[coordinate](loop_4_fork_angle_r) at (\x1+10mm, \y2+0mm) {};
\path[-](loop_4_fork) edge [] node [above]{$yes$} (loop_4_fork_angle_r);
\node[below=of loop_4_fork_angle_r.south, rectangle, draw, yshift=-3mm](assignment_5){$x = x+3+a-b$};
\path[->](loop_4_fork_angle_r) edge [] node [right]{} (assignment_5);
\node[below=of assignment_5.south, rectangle, draw, yshift=3mm](assignment_6){$a = (a-1)*(a+2)$};
\path[->](assignment_5) edge [] node [right]{} (assignment_6);
\node[below=of assignment_6.south, rectangle, draw, yshift=3mm](assignment_7){$b = -b+3$};
\path[->](assignment_6) edge [] node [right]{} (assignment_7);
\draw let \p1=(assignment_7.south) in let \p2=(assignment_7.south) in node[coordinate](loop_4_back_edge_angle1) at (\x1+0mm, \y2+-6mm) {};
\path[-](assignment_7) edge [] node [right]{} (loop_4_back_edge_angle1);
\draw let \p1=(loop_4_back_edge_angle1) in let \p2=(loop_4_back_edge_angle1) in node[coordinate](loop_4_back_edge_angle2) at (\x1+25mm, \y2+0mm) {};
\path[-](loop_4_back_edge_angle1) edge [] node [right]{} (loop_4_back_edge_angle2);
\draw let \p1=(loop_4_back_edge_angle2) in let \p2=(loop_4_join) in node[coordinate](loop_4_back_edge_angle3) at (\x1+0mm, \y2+0mm) {};
\path[-](loop_4_back_edge_angle2) edge [] node [right]{} (loop_4_back_edge_angle3);
\path[->](loop_4_back_edge_angle3) edge [] node [right]{} (loop_4_join);
\node[below=of loop_4_fork_angle_l.south, circle, draw,yshift=-3mm](stop){Stop};
\path[->](loop_4_fork_angle_l) edge [] node [left]{} (stop);

		\end{tikzpicture}
	\end{center}

	\noindent\hint{Finding a formula for $r$ is non-trivial in this program. Make sure you understand what the program is doing and consider writing down the values of variables for a couple of iterations.}
	
	\noindent\hint{Keep in mind, that $r$ need not necessarily represent the exact number of remaining loop iterations, but a finite upper limit.}

\end{assignment}

\begin{comment}
\begin{assignment}[H,points=8]{Name}
TODO: fix, this does not terminate for negative n
	\begin{center}
		\begin{tikzpicture}
			\node[circle,draw](start){Start};
\node[below=of start.south, rectangle, draw, yshift=3mm](assignment_0){$x = 1$};
\path[->](start) edge [] node [right]{} (assignment_0);
\node[below=of assignment_0.south, rectangle, draw, yshift=3mm](assignment_1){$i = 0$};
\path[->](assignment_0) edge [] node [right]{} (assignment_1);
\node[below=of assignment_1.south, rectangle, draw, yshift=3mm](read_2){$n = read()$};
\path[->](assignment_1) edge [] node [right]{} (read_2);
\node[below=of read_2.south, circle, draw, yshift=3mm](loop_3_join){};
\path[->](read_2) edge [] node [right]{} (loop_3_join);
\node[below=of loop_3_join.south, diamond, draw, yshift=3mm](loop_3_fork){i == n};
\path[->](loop_3_join) edge [] node [right]{} (loop_3_fork);
\draw let \p1=(loop_3_fork.west) in let \p2=(loop_3_fork.west) in node[coordinate](loop_3_fork_angle_l) at (\x1+-10mm, \y2+0mm) {};
\path[-](loop_3_fork) edge [] node [above]{$yes$} (loop_3_fork_angle_l);
\draw let \p1=(loop_3_fork.east) in let \p2=(loop_3_fork.east) in node[coordinate](loop_3_fork_angle_r) at (\x1+10mm, \y2+0mm) {};
\path[-](loop_3_fork) edge [] node [above]{$no$} (loop_3_fork_angle_r);
\node[below=of loop_3_fork_angle_l.south, rectangle, draw, yshift=3mm](write_4){$write(x)$};
\path[->](loop_3_fork_angle_l) edge [] node [left]{} (write_4);
\node[below=of loop_3_fork_angle_r.south, rectangle, draw, yshift=3mm](assignment_5){$i = i+1$};
\path[->](loop_3_fork_angle_r) edge [] node [right]{} (assignment_5);
\node[below=of assignment_5.south, rectangle, draw, yshift=3mm](assignment_6){$j = i$};
\path[->](assignment_5) edge [] node [right]{} (assignment_6);
\node[below=of assignment_6.south, circle, draw, yshift=3mm](loop_7_join){};
\path[->](assignment_6) edge [] node [right]{} (loop_7_join);
\node[below=of loop_7_join.south, diamond, draw, yshift=3mm](loop_7_fork){j > 0};
\path[->](loop_7_join) edge [] node [right]{} (loop_7_fork);
\draw let \p1=(loop_7_fork.west) in let \p2=(loop_7_fork.west) in node[coordinate](loop_7_fork_angle_l) at (\x1+-10mm, \y2+0mm) {};
\path[-](loop_7_fork) edge [] node [above]{$no$} (loop_7_fork_angle_l);
\draw let \p1=(loop_7_fork.east) in let \p2=(loop_7_fork.east) in node[coordinate](loop_7_fork_angle_r) at (\x1+10mm, \y2+0mm) {};
\path[-](loop_7_fork) edge [] node [above]{$yes$} (loop_7_fork_angle_r);
\node[below=of loop_7_fork_angle_r.south, rectangle, draw, yshift=3mm](assignment_8){$x = x*i$};
\path[->](loop_7_fork_angle_r) edge [] node [right]{} (assignment_8);
\node[below=of assignment_8.south, rectangle, draw, yshift=3mm](assignment_9){$j = j-1$};
\path[->](assignment_8) edge [] node [right]{} (assignment_9);
\draw let \p1=(assignment_9.south) in let \p2=(assignment_9.south) in node[coordinate](loop_7_back_edge_angle1) at (\x1+0mm, \y2+-6mm) {};
\path[-](assignment_9) edge [] node [right]{} (loop_7_back_edge_angle1);
\draw let \p1=(loop_7_back_edge_angle1) in let \p2=(loop_7_back_edge_angle1) in node[coordinate](loop_7_back_edge_angle2) at (\x1+14mm, \y2+0mm) {};
\path[-](loop_7_back_edge_angle1) edge [] node [right]{} (loop_7_back_edge_angle2);
\draw let \p1=(loop_7_back_edge_angle2) in let \p2=(loop_7_join) in node[coordinate](loop_7_back_edge_angle3) at (\x1+0mm, \y2+0mm) {};
\path[-](loop_7_back_edge_angle2) edge [] node [right]{} (loop_7_back_edge_angle3);
\path[->](loop_7_back_edge_angle3) edge [] node [right]{} (loop_7_join);
\draw let \p1=(loop_7_fork_angle_l.south) in let \p2=(loop_7_fork_angle_l.south) in node[coordinate](loop_3_back_edge_angle1) at (\x1+0mm, \y2+-40mm) {};
\path[-](loop_7_fork_angle_l) edge [] node [left]{} (loop_3_back_edge_angle1);
\draw let \p1=(loop_3_back_edge_angle1) in let \p2=(loop_3_back_edge_angle1) in node[coordinate](loop_3_back_edge_angle2) at (\x1+62mm, \y2+0mm) {};
\path[-](loop_3_back_edge_angle1) edge [] node [right]{} (loop_3_back_edge_angle2);
\draw let \p1=(loop_3_back_edge_angle2) in let \p2=(loop_3_join) in node[coordinate](loop_3_back_edge_angle3) at (\x1+0mm, \y2+0mm) {};
\path[-](loop_3_back_edge_angle2) edge [] node [right]{} (loop_3_back_edge_angle3);
\path[->](loop_3_back_edge_angle3) edge [] node [right]{} (loop_3_join);
\node[below=of write_4.south, circle, draw,yshift=3mm](stop){Stop};
\path[->](write_4) edge [] node [left]{$\mbox{Z} :\equiv x = {\displaystyle \prod_{k = 1}^{n} (k)^{k}}$} (stop);

		\end{tikzpicture}
	\end{center}
\end{assignment}
\end{comment}

\end{document}





































